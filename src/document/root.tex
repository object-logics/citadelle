\documentclass[11pt,a4paper,openright,twoside,abstracton]{scrreprt}
\usepackage{isabelle,isabellesym}
\usepackage[nocolortable, noaclist]{hol-ocl-isar}
\usepackage{fmde-acronyms}
\usepackage{paralist}
\usepackage{booktabs}
\usepackage{graphicx}
\usepackage{amssymb}
\usepackage[numbers, sort&compress, sectionbib]{natbib}
\usepackage[caption=false]{subfig}
\usepackage{lstisar}
\usepackage{paralist}
\usepackage[]{mathtools}
\usepackage[pdfpagelabels, pageanchor=false, plainpages=false]{hyperref}


\renewcommand{\HolTrue}{\mathrm{true}} 
\renewcommand{\HolFalse}{\mathrm{false}}


\newcommand{\testgen}{\acs{hol}-TestGen\xspace} 
\DeclareMathOperator{\vars}{\mathit{vars}}
\renewcommand{\to}{\Rightarrow}
\newcommand{\HolRef}{\mathrm{ref}}
\newcommand{\HolOption}{\mathrm{option}}
\newcommand{\term}{\mathrm{term}}
\newcommand{\ran}{\mathrm{ran}}
\newcommand{\dom}{\mathrm{dom}} 
\newcommand{\ind}{\mathrm{ind}}
\newcommand{\order}{\mathrm{order}}
\newcommand{\types}{\mathrm{types}}
\newcommand{\typedef}{\mathrm{typedef}}
\DeclareMathOperator{\upCase}{upCase}
\DeclareMathOperator{\optionCase}{optionCase}
\DeclareMathOperator{\image}{image}
\DeclareMathOperator{\fix}{fix}

\newcommand{\mi}[1]{\,\text{#1}}
\newcommand{\state}[1]{\ifthenelse{\equal{}{#1}}%
  {\operatorname{state}}%
  {\operatorname{\mathit{state}}(#1)}%
}

\renewcommand{\ofType}{\mathbin{::}}
\newcommand{\mocl}[1]{\text{\inlineocl{#1}}}
\DeclareMathOperator{\TCnull}{null}
\DeclareMathOperator{\HolNull}{null}
\DeclareMathOperator{\HolBot}{bot}

\graphicspath{{data/},{figures/}}

\newtheorem{example}{Example}


% urls in roman style, theory text in math-similar italics
\urlstyle{rm}
\isabellestyle{it}
\newcommand{\ie}{i.\,e.\xspace}
\newcommand{\eg}{e.\,g.\xspace}
\renewcommand{\isamarkupheader}[1]{\chapter{#1}}
\renewcommand{\isamarkupsection}[1]{\section{#1}}
\renewcommand{\isamarkupsubsection}[1]{\subsection{#1}}
\renewcommand{\isamarkupsubsubsection}[1]{\subsubsection{#1}}
\renewcommand{\isamarkupsect}[1]{\section{#1}}
\renewcommand{\isamarkupsubsect}[1]{\susubsection{#1}}
\renewcommand{\isamarkupsubsubsect}[1]{\subsubsection{#1}}

\begin{document}

\title{Featherweight OCL}
\subtitle{A Study for a Consistent Semantics of UML/OCL 2.3 in HOL}
\subject{Extended Version}
\author{Achim D. Brucker \and Burkhart Wolff}


\maketitle

\begin{abstract}
  At its origins, \OCL was conceived as a strict semantics for
  undefinedness, with the exception of the logical connectives of type
  \inlineocl{Boolean} that constitute a three-valued propositional
  logic.  Recent versions of the \OCL standard added a second
  exception element, which, similar to the null references in
  programming languages, is given a non-strict semantics.

  In this paper, we report on our results in formalizing the core of
  \acs{ocl} in \acf{hol}. This formalization revealed several
  inconsistencies and contradictions in the current version of the
  \acs{ocl} standard.  These inconsistencies and contradictions are
  reflected in the challenge to define and implement \OCL tools in a
  uniform manner.


  
  \vspace{1cm}\noindent{\textbf{Further readings:}} This theory extends the paper
  ``Featherweight OCL: A study for the consistent semantics of \OCL 2.3
  in \HOL''~\cite{brucker.ea:featherweight:2012} that is published as
  part of the proceedings of the \OCL workshop 2012.
\end{abstract}



\tableofcontents

%%%%%%%%%%%%%%%%%%%%%%%%%%%%%%%%%%%%%%%%%%%%%%%%%%%%%%%%%%%%%%%%%%%
% TODO: 
%%%%%%%
% 1) Motivation: Achim: Deadline Monday
%   - motivation with respect to history/glitches
%   - mention aachen meeting and RFP (Machine-generated document)
%   - include intro to OCL introudctioin
%%%%%%%%%%%%%%%%%%%%%%%%%%%%%%%%%%%%%%%%%%%%%%%%%%%%%%%%%%%%%%%%%%%%
\part{Introduction}
\chapter{Motivation}
The Unified Modeling Language
(UML)~\cite{omg:uml-infrastructure:2011,omg:uml-superstructure:2011}
is one of the few modeling languages that is widely used in
industry. While UML is mostly known as diagrammatic modeling language
(\eg, visualizing class models), it also comprises a textual language,
called Object Constraint Language (OCL)~\cite{omg:ocl:2012}. OCL is a
textual annotation language, originally based on a three-valued logic, that 
turns substantial parts of UML into a formal language.  Unfortunately the semantics of this
specification language, captured in the ``Annex A'' (originally, based
on the work of Richters~\cite{richters:precise:2002}) of the OCL
standard leads to different interpretations of corner cases.  Many of
these corner cases had been subject to formal analysis since more than
nearly fifteen years (see,
\eg,~\cite{brucker.ea:semantic:2006-b,brucker.ea:proposal:2002,mandel.ea:ocl:1999,
hamie.ea:reflections:1998,cook.ea::amsterdam:2002}).

At its origins~\cite{richters:precise:2002,omg:ocl:1997}, OCL was
conceived as a strict semantics for undefinedness, with the exception
of the logical connectives of type \inlineocl{Boolean} that constitute
a three-valued propositional logic. At its core, OCL comprises three
layers: 
\begin{enumerate}
\item Operators (\eg, \inlineocl{_ and _}, \inlineocl{_ + _}) on
  built-in data structures such as \inlineocl{Boolean},
  \inlineocl{Integer}, or typed sets (\inlineocl{Set(_)}.
\item Operators on the user-defined data model (\eg, defined in terms
  of an UML class model) such as accessors, type casts and tests.
\item Arbitrary, user-defined, side-effect-free methods.
\end{enumerate}

Motivated by the need for aligning OCL closer with UML, recent
versions of the OCL standard~\cite{omg:ocl:2006,omg:ocl:2012} added a
second exception element.  While the first exception element
\mocl{undefined} has a strict semantics, \mocl{null} has a non strict
semantic interpretation.  Unfortunately, this extension results in
several inconsistencies and contradictions. These problems are
reflected in difficulties to define interpreters, code-generators,
specification animators or theorem provers for OCL in a uniform
manner and resulting incompatibilities of various tools.

For the OCL community, the \mocl{undefined}/\mocl{null} and many related
issues results in the challenge to define a
consistent version of the OCL standard that is well aligned with the
recent developments of the UML. A syntactical and semantical
consistent standard requires a major revision of both the informal and
formal parts of the standard. To discuss the future directions of the
standard, a number of OCL experts met in November 2013 in
Aachen to discuss possible mid-term term improvements of OCL,
strategies of standardization of OCL within the OMG  
and a vision for possible long-term developments of the
language~\cite{brucker.ea:summary-aachen:2013}. During this meeting,
a Request for Proposals (RFP) for OCL 2.5 was finalized and meanwhile
prosed. In particular, this RFP requires that the future OCL 2.5 standard 
document shall be generated from a machine-checked source. This will ensure 
\begin{itemize}
\item the absence of syntax errors and
\item the consistency of the formal semantics,
\item a suite of corner-cases relevant for OCL tool implementors.
\end{itemize}

In this document, we present a formalization using
Isabelle/HOL~\cite{nipkow.ea:isabelle:2002} of a core language of
OCL. The semantic theory, based on a so-called ``shallow embedding'',
is called \emph{Featherweight OCL}, since it focusses on a formal treatment
of the key-elements of the language (rather than a full treatment
  of all operators and thus, a ``complete'' implementation). In contrast to  full OCL,
it comprises just the logic captured in \verb+Boolean+,  
the basic data type \verb+Integer+, the collection type \verb+Set+, 
as well as the generic construction principle of Class Models,
which is instantiated and demonstrated for two examples
(an automated support for this type-safe construction is again out of the
  scope of Featherweight OCL).
This formal semantics definition
is intended to be a proposal for the standardization process of 
OCL 2.5, which should ultimately replace parts of the mandatory
part of the standard document\cite{omg:ocl:2012} as well as replace completely
its  informative ``Annex A''.


\chapter{Background}
\section{A guided tour through UML/OCL}
\label{sec:guidedtour}
The Unified Modeling Language (UML) comprises a variety of model types
for describing static (\eg, class models, object models) and dynamic
(\eg, state-machines, activity graphs) system properties.
\begin{figure*}
  \centering\scalebox{1}{\includegraphics{figures/AbstractSimpleChair}}%
  \caption{A simple UML class model representing a conference
    system for organizing conference sessions: persons can
    participate, in different roles, in a session. \label{fig:uml}}
\end{figure*}
One of the more prominent model types of the UML is the
\emph{class model} (visualized as \emph{class diagram}) for modeling
the underlying data model of a system in an object-oriented manner. As
a running example, we model a part of a conference management
system. Such a system usually supports the conference organizing
process, \eg, creating a conference Website, reviewing submissions,
registering attendees, organizing the different sessions and tracks,
and indexing and producing the resulting proceedings. In this example,
we constrain ourselves to the process of organizing conference
sessions; \autoref{fig:uml} shows the class model.  We model the
hierarchy of roles of our system as a hierarchy of classes (\eg,
\inlineocl{Hearer}, \inlineocl{Speaker}, or \inlineocl{Chair}) using
an \emph{inheritance} relation (also called \emph{generalization}). In
particular, \emph{inheritance} establishes a \emph{subtyping}
relationship, \ie, every \inlineocl{Speaker} (\emph{subclass}) is also
a \inlineocl{Hearer} (\emph{superclass}).

A class does not only describe a set of \emph{instances} (called
\emph{objects}), \ie, record-like data consisting of \emph{attributes}
such as \inlineocl{name} of class \inlineocl{Session}, but also
\emph{operations} defined over them. For example, for the class
\inlineocl{Session}, representing a conference session, we model an
operation \inlineocl{findRole(p:Person):Role} that should return the
role of a \inlineocl{Person} in the context of a specific session;
later, we will describe the behavior of this operation in more detail
using UML. In the following, the term object describes a
(run-time) instance of a class or one of its subclasses.

Relations between classes (called \emph{associations} in UML)
can be represented in a class diagram by connecting lines, \eg,
\inlineocl{Participant} and \inlineocl{Session} or \inlineocl{Person}
and \inlineocl{Role}. Associations may be labeled by a particular
constraint called \emph{multiplicity}, \eg, \inlineocl+0..*+ or
\inlineocl+0..1+, which means that in a relation between participants
and sessions, each \inlineocl{Participant} object is associated to at
most one \inlineocl{Session} object, while each \inlineocl{Session}
object may be associated to arbitrarily many \inlineocl{Participant}
objects. Furthermore, associations may be labeled by projection
functions like \inlineocl{person} and \inlineocl{role}; these implicit
function definitions allow for OCL-expressions like
\inlineocl+self.person+, where \inlineocl+self+ is a variable of the
class \inlineocl{Role}. The expression \inlineocl+self.person+ denotes
persons being related to the a specific object \inlineocl{self} of
type role. A particular feature of the UML are \emph{association
  classes} (\inlineocl{Participant} in our example) which represent a
concrete tuple of the relation within a system state as an object;
\ie, associations classes allow also for defining attributes and
operations for such tuples. In a class diagram, association classes
are represented by a dotted line connecting the class with the
association. Associations classes can take part in other associations.
Moreover, UML supports also $n$-ary associations (not shown in
our example).

We refine this data model using the Object Constraint Language (OCL)
for specifying additional invariants, preconditions and postconditions
of operations. For example, we specify that objects of the class
\inlineocl{Person} are uniquely determined by the value of the
\inlineocl{name} attribute and that the attribute \inlineocl{name} is
not equal to the empty string (\inlineocl{''}):
\fixme{notation: should we use our notation or the notation of the standard ?}
\begin{ocl}
context Person
  inv: name $\isasymMathOclStrictNotEq$ '' $\isasymMathOclAnd$
       Person::allInstances()->isUnique(p:Person | p.name)
\end{ocl}
Moreover, we specify that every session has exactly one chair by the
following invariant (called \inlineocl{onlyOneChair}) of the class
\inlineocl{Session}:
\begin{ocl}
context Session
  inv onlyOneChair:
      self.participants->one( p:Participant | $\isType{Chair}{(\mathtt{p.role})}$)
\end{ocl}
where $\isType{Chair}{(\mathtt{p.role})}$ evaluates to true, if
\inlineocl{p.role} is of \emph{dynamic type}
\inlineocl{Chair}. Besides the usual \emph{static types} (\ie, the
types inferred by a static type inference), objects in UML and other
object-oriented languages have a second \emph{dynamic} type concept.
This is a consequence of a family of \emph{casting functions} (written
$\typeCast{o}{C}$ for an object $o$ into another class type $C$) that
allows for converting the static type of objects along the class
hierarchy. The dynamic type of an object can be understood as its
``initial static type'' and is unchanged by casts. We complete our
example by describing the behavior of the operation
\inlineocl{findRole} as follows:
\begin{ocl}
context Session::findRole(person:Person):Role
  pre:  person $\isasymMathOclIncludes$ self.participates.person
  post: result=self.participants->one(p:Participant |
                                p.person $\isasymMathOclStrictEq$ person ).role
        $\isasymMathOclAnd$ self.participants $\isasymMathOclStrictEq$ self.participants@pre
        $\isasymMathOclAnd$ self.name $\isasymMathOclStrictEq$ self.name@pre
\end{ocl}
where in post-conditions, the operator \inlineocl{@pre} allows for
accessing the previous state.

In UML, classes can contain attributes of the type of the
defining class.  Thus, UML can represent (mutually) recursive
datatypes. Moreover, OCL introduces also recursively specified
operations.

A key idea of defining the semantics of UML and extensions like
SecureUML~\cite{brucker.ea:transformation:2006} is to translate the
diagrammatic UML features into a combination of more elementary
features of UML and OCL
expressions~\cite{gogolla.ea:expressing:2001}. For example,
associations are usually represented by collection-valued class
attributes together with OCL constraints expressing the
multiplicity. Thus, having a semantics for a subset of UML and OCL is
tantamount for the foundation of the entire method.






\section{Formal Foundation}


\subsection{Isabelle}
Isabelle~\cite{nipkow.ea:isabelle:2002} is a \emph{generic} theorem
prover. New object logics can be introduced by specifying their syntax
and natural deduction inference rules. Among other logics, Isabelle
supports first-order logic, Zermelo-Fraenkel set theory and the
instance for Church's higher-order logic (HOL), which we choose as
basis for \testgen and which is introduced in the subsequent section.

Isabelle's inference rules are based on the built-in meta-level
implication $\_ \Implies \_$ allowing to form constructs like $A_1
\Implies \cdots \Implies A_n \Implies A_{n+1}$, which are viewed as a
\emph{rule} of the form ``from assumptions $A_1$ to $A_n$, infer
conclusion $A_{n+1}$'' and which is written in Isabelle as
\begin{gather}
  \semantics{A_1 ; \ldots; A_n}\Implies A_{n+1}
  \qquad
  \text{or, in mathematical notation,}
  \qquad
  \begin{prooftree}
    A_1 \qquad \cdots \qquad A_n
    \justifies
    A_{n+1}
    \ptmi{.}
  \end{prooftree}
\end{gather}
The built-in meta-level quantification $\Forall x\spot  x$ captures
the usual side-constraints ``$x$ must not occur free in the
assumptions'' for quantifier rules; meta-quantified variables can be
considered as ``fresh'' free variables. Meta-level quantification
leads to a generalization of Horn-clauses of the form:
\begin{gather}
\Forall x_1, \ldots, x_m\spot \semantics{A_1 ; \ldots; A_n}\Implies
A_{n+1}\mi{.}
\end{gather}

Isabelle supports forward- and backward reasoning on rules.  For
backward-reasoning, a \emph{proof-state} can be initialized and
further transformed into others. For example, a proof of $\phi$, using
the Isar~\cite{wenzel:isabelleisar:2002} language, will look as
follows in Isabelle:
\begin{gather}
  \begin{array}{l}
    \Lemma{label} \phi\\
    \quad\apply{case\_tac}\\
    \quad\apply{simp\_all}\\
  \done
  \end{array}
\end{gather}
This proof script instructs Isabelle to prove $\phi$ by case
distinction followed by a simplification of the resulting proof state.
Such a proof state is an implicitly conjoint sequence of generalized
Horn-clauses (called \emph{subgoals}) $\phi_1$, \ldots,$\phi_n$ and a
\emph{goal} $\phi$. Proof states were usually denoted by:
\begin{gather}
\begin{array}{rl}
\pglabel{label}:& \phi \\
 1.& \phi_1 \\
    &\vdots \\
 n.& \phi_n\\
\end{array}
\end{gather}
Subgoals and goals may be extracted from the proof state into theorems
of the form $\semantics{\phi_1 ; \ldots; \phi_n}\Implies \phi$ at any
time; this mechanism helps to generate test theorems.  Further,
Isabelle supports meta-variables (written $\meta{x}, \meta{y},
\ldots$), which can be seen as ``holes in a term'' that can still be
substituted. Meta-variables are instantiated by Isabelle's built-in
higher-order unification.

\subsection{Higher-order logic}
\emph{Higher-order logic}
(HOL)~\cite{church:types:1940,andrews:introduction:2002} is a
classical logic based on a simple type system.  It provides the usual
logical connectives like $\_ \land \_$, $\_ \implies\_$, $\lnot \_ $
as well as the object-logical quantifiers $\forall x\spot P\ap x$ and
$\exists x\spot P\ap x$; in contrast to first-order logic, quantifiers
may range over arbitrary types, including total functions
$f\ofType\alpha \Rightarrow \beta$. HOL is centered around
extensional equality $\_ = \_ \ofType \alpha \Rightarrow \alpha
\Rightarrow \text{bool}$.  HOL is more expressive than first-order
logic, since, \eg, induction schemes can be expressed inside the
logic. Being based on some polymorphically typed $\lambda$-calculus,
HOL can be viewed as a combination of a programming language
like SML or Haskell and a specification language providing
powerful logical quantifiers ranging over elementary and function
types.

Isabelle/HOL is a logical embedding of HOL into Isabelle.  The
(original) simple-type system underlying HOL has been extended by
Hindley-Milner style polymorphism with type-classes similar to
Haskell.  While Isabelle/HOL is usually seen as proof assistant, we
use it as symbolic computation environment. Implementations on top of
Isabelle/HOL can re-use existing powerful deduction mechanisms such
as higher-order resolution, tableaux-based reasoners, rewriting
procedures, Presburger arithmetic, and via various integration
mechanisms, also external provers such as Vampire and the
SMT-solver Z3.

Isabelle/HOL offers support for a particular methodology to extend
given theories in a logically safe way: A theory-extension is
\emph{conservative} if the extended theory is consistent provided that
the original theory was consistent.  Conservative extensions can be
\emph{constant definitions}, \emph{type definitions}, \emph{datatype
  definitions}, \emph{primitive recursive definitions} and
\emph{wellfounded recursive definitions}.

For instance, the library includes the type constructor $\up{\tau} :=
\isasymbottom ~ | ~ \lift{\_} : \alpha$ that assigns to each type
$\tau$ a type $\up{\tau}$ \emph{disjointly extended} by the
exceptional element $\isasymbottom$\index{undefinedness lifting}. The
function $\drop{\_} : \up{\alpha} \to \alpha$ is the inverse of
$\lift{\_}$ (unspecified for $\isasymbottom$). Partial functions
$\alpha \isasymrightharpoonup \beta$ are defined as functions $\alpha
\isasymRightarrow \up{\beta}$ supporting the usual concepts of domain
($\dom\;\_$) and range ($\ran\;\_$).

As another example of a conservative extension, typed sets were built
in the Isabelle libraries conservatively on top of the kernel of HOL
as functions to $\HolBoolean$; consequently, the constant definitions
for membership is as follows:\footnote{To increase readability, we use
  a slightly simplified presentation.}
\begin{gather}
  \begin{array}{lrll}
    \types& \alpha \HolSet            &= \alpha \Rightarrow \HolBoolean\\[.5ex]
    \definitionS &\operatorname{Collect}&\ofType (\alpha \Rightarrow
     \HolBoolean) \Rightarrow \HolSet{\alpha}  &\qquad\text{--- set comprehension}\\
    \where &\operatorname{Collect}\ap S      &\equiv S\\[.5ex]
     \definitionS &\operatorname{member}           &\ofType \alpha \Rightarrow
     \alpha \Rightarrow \HolBoolean &\qquad\text{---
       membership test}\\
     \where &\operatorname{member}\ap s\ap S &\equiv S s\\
  \end{array}
\end{gather}
Isabelle's syntax engine is instructed to accept the notation
$\{x \mid P\}$ for $\operatorname{Collect}\ap\lambda x\spot P$ and the
notation $s \in S$ for $\operatorname{member}\ap s\ap S$. As can be
inferred from the example, constant definitions are axioms that
introduce a fresh constant symbol by some closed, non-recursive
expressions; this type of axiom is logically safe since it works
like an abbreviation. The syntactic side conditions of this axiom are
mechanically checked, of course. It is straightforward to express the
usual operations on sets like $\_ \cup \_,
\_\cap\_\ofType\HolSet{\alpha} \Rightarrow \HolSet{\alpha} \Rightarrow
\HolSet{\alpha}$ as conservative extensions, too, while the rules of
typed set theory were derived by proofs from these definitions.

Similarly, a logical compiler is invoked for the following statements introducing
the types option and list:
\begin{gather}
  \begin{array}{lrll}
    \datatype & \HolOption       &= \HolNone \mid \HolSome{\alpha}\\[.5ex]
    \datatype & \HolList{\alpha} &= \operatorname{Nil} \mid
    \operatorname{Cons}\ap a\ap l
  \end{array}
\end{gather}
Here, $[]$ or $a\#l$ are an alternative syntax for $\operatorname{Nil}$
or $\operatorname{Cons}\ap a ~l$; moreover, $[a, b, c]$ is defined as
alternative syntax for $a\#b\#c\#[]$. These (recursive) statements
were internally represented in by internal type and constant
definitions. Besides the \emph{constructors} $\HolNone$, $\HolSome$,
$[]$ and $\operatorname{Cons}$, there is the match operation
\begin{gather}
\HolCase\ap x\ap\HolOf~\HolNone \isasymRightarrow F\ap \mid
\HolSome{a} \isasymRightarrow G\ap a
\end{gather}
respectively
\begin{gather}
\HolCase\ap x\ap\HolOf~[] \isasymRightarrow F\ap \mid \operatorname{Cons}\ap a\ap
r \isasymRightarrow G\ap a\ap r\mi{.}
\end{gather}
From the internal definitions (not shown here) a number of properties
were automatically derived. We show only the case for lists:
\begin{gather}\label{eq:datatype-rules}
  \begin{array}{ll}
    (\HolCase\ap[]\ap\HolOf\ap[] \Rightarrow F  \ap | \ap  (a\#r) \Rightarrow
    G\ap a\ap r) = F &\\
    (\HolCase \ap  b\#t  \ap \HolOf  \ap [] \Rightarrow F  \ap  | \ap
    (a\#r) \Rightarrow G\ap a\ap r) = G~b~t &\\ %
    \mbox{}[] \neq a\#t    &\text{-- distinctness} \\
    \semantics{ a = [] \implies P ; \exists~x~t\spot  a = x\#t \implies P } \Longrightarrow P &\text{-- exhaust} \\
    \semantics{ P [] ; \forall~at\spot  P t \implies P (a\#t) } \Longrightarrow P x      &\text{-- induct}
  \end{array}
\end{gather}
Finally, there is a compiler for primitive and wellfounded recursive
function definitions. For example, we may define the
$\operatorname{sort}$ operation of our running test example by:
\begin{gather}\label{eq:sortdef}
  \begin{array}{lll}
    \fun
    &\enspace\operatorname{ins} & \ofType
    [\alpha\ofType\mathrm{linorder}, \HolList{\alpha}]
    \Rightarrow
    \HolList{\alpha}\\
    \where
    &\enspace \operatorname{ins}\ap x \ap  [\;] &= [x]\\
    &\enspace \operatorname{ins}\ap x \ap (y\#\mathit{ys})&=
    \HolIf x < y
    \HolThen x\#  y \# ys
    \HolElse y\#(\operatorname{ins} \ap x \ap ys)
 \end{array}\\
  \begin{array}{lll}
    \fun
    &\enspace\operatorname{sort} & \ofType
    \HolList{(\alpha\ofType\mathrm{linorder})}
    \Rightarrow
    \HolList{\alpha}\\
    \where
    &\enspace \operatorname{sort}\ap [\;] &= [\;]\\
    &\enspace \operatorname{sort} (x\#\mathit{xs})&=
    \operatorname{ins}\ap x\ap (\operatorname{sort}\ap xs)
   \end{array}
\end{gather}
The internal (non-recursive) constant definition for these operations
is quite involved; however, the logical compiler will finally derive
all the equations in the statements above from this definition and
make them available for automated simplification.

Thus, Isabelle/HOL also provides a large collection of theories like
sets, lists, multisets, orderings, and various arithmetic theories
which only contain rules derived from conservative definitions. In
particular, Isabelle manages a set of \emph{executable types and
  operators}, \ie, types and operators for which a compilation to
SML, OCaml or Haskell is possible. Setups for arithmetic types
such as $\text{int}$ have been done; moreover any datatype and any
recursive function were included in this executable set (providing
that they only consist of executable operators). Similarly, Isabelle
manages a large set of (higher-order) rewrite rules into which
recursive function definitions were included. Provided that this
rule set represents a terminating and confluent rewrite system, the
Isabelle simplifier provides also a highly potent decision procedure
for many fragments of theories underlying the constraints to be
processed when constructing test theorems.







% \subsection{Higher-order Logic and Isabelle}\label{sec:higher-order-logic}
% Higher-order Logic
% (\acs{hol})~\cite{church:types:1940,andrews:introduction:2002} is a
% classical logic with equality enriched by total polymorphic
% higher-order functions. It is more expressive than first-order logic,
% \eg, induction schemes can be expressed inside the
% logic. Pragmatically, HOL can be viewed as ``Haskell with
% Quantifiers.''

% HOL is based on the typed $\lambda$-calculus, \ie, the \emph{terms} of HOL are
% $\lambda$-expressions.  Types of terms may be built from \emph{type
%   variables}\index{type variables} (like $\alpha$, $\beta$, \ldots, optionally
% annotated by Haskell-like \emph{type classes}\index{type class} as in
% $\alpha::order$ or $\alpha::\Bot$) or \emph{type constructors}\index{type
%   constructor}. Type constructors may have arguments (as in $\HolList{\alpha}$
% or $\HolSet{\alpha}$). The type constructor for the function space
% $\isasymRightarrow$ is written infix: $\alpha \isasymRightarrow \beta$; multiple
% applications like
% $\tau_1\isasymRightarrow(\ldots\isasymRightarrow(\tau_n\isasymRightarrow\tau_{n+1})\ldots)$
% have the alternative syntax $[\tau_1, \ldots, \tau_n] \isasymRightarrow
% \tau_{n+1}$. HOL is centered around the extensional logical equality $\_=\_$
% with type $[\alpha, \alpha] \isasymRightarrow \HolBoolean$, where $\HolBoolean$
% is the fundamental logical type.  We use infix notation: instead of
% $(\_=\_)~E_1~E_2$ we write $E_1=E_2$. The logical connectives $\_\isasymand\_$,
% $\_\isasymor\_$, $\_\to\_$ of HOL have type $[\HolBoolean, \HolBoolean]
% \isasymRightarrow \HolBoolean$, $\isasymnot\_$ has type $\HolBoolean
% \isasymRightarrow \HolBoolean$. The quantifiers $\isasymforall\_.\_$ and
% $\isasymexists\_.\_$ have type $[\alpha \isasymRightarrow \HolBoolean]
% \isasymRightarrow \HolBoolean$. The quantifiers may range over types of higher
% order, \ie, functions or sets.
% The definition of the element-hood $\_\isasymin\_$, the set comprehension
% $\{\_.\_\}$, as well as $\_\isasymunion\_$ and $\_\isasyminter\_$ are standard.

% Isabelle is a theorem is generic interactive theorem proving system;
% Isabelle/HOL is an instance of the former with HOL.  The
% Isabelle/HOL library contains formal definitions and theorems for a
% wide range of mathematical concepts used in computer science,
% including typed set theory, well-founded recursion theory, number
% theory and theories for data-structures like Cartesian products
% $\alpha \times \beta$ and disjoint type sums $\alpha + \beta$.
% The library also includes the type constructor $\up{\tau} := \isasymbottom
% ~ | ~ \lift{\_} : \alpha$ that assigns to each type $\tau$ a type
% $\up{\tau}$ \emph{disjointly extended} by the exceptional element
% $\isasymbottom$\index{undefinedness lifting}. The function $\drop{\_}
% : \up{\alpha} \to \alpha$ is the inverse of $\lift{\_}$ (unspecified
% for $\isasymbottom$). Partial functions $\alpha \isasymrightharpoonup
% \beta$ are defined as functions $\alpha \isasymRightarrow \up{\beta}$
% supporting the usual concepts of domain ($\dom\;\_$) and range
% ($\ran\;\_$).
% The library is built entirely by logically safe,
% conservative definitions and derived rules.  This methodology is also
% applied to \holocl~\cite{brucker.ea:hol-ocl:2008} and Featherweight
% OCL.

% For Journal - or long- version


\section{Featherweight OCL: Design Goals}
% generated text of all theories
Featherweight OCL is a formalization of the core of OCL
aiming at formally investigating the relationship between the
different notions of ``undefinedness,'' \ie, \inlineocl{invalid} and

\inlineocl{null}. As such, it does not attempt to define the complete
OCL library. Instead, it concentrates on the core concepts of
OCL as well as the types \inlineocl{Boolean},
\inlineocl{Integer}, and typed sets (\inlineocl|Set(T)|).  Following
the tradition of
\holocl~\cite{brucker.ea:hol-ocl:2008,brucker.ea:hol-ocl-book:2006},
Featherweight OCL is based on the following principles:
\begin{enumerate}
\item It is an embedding into a powerful semantic meta-language and
  environment, namely
  Isabelle/HOL~\cite{nipkow.ea:isabelle:2002}.
\item It is a \emph{shallow embedding} in HOL; types
  in OCL were injectively mapped to types in Featherweight
  OCL. Ill-typed OCL specifications cannot be represented in
  Featherweight OCL and a type in Featherweight OCL contains exactly
  the values that are possible in OCL. Thus, sets may contain
  \inlineocl{null} (\inlineocl|Set{null}| is a defined set) but not
  \inlineocl{invalid} (\inlineocl|Set{invalid}| is just
  \inlineocl{invalid}).
\item Any Featherweight OCL type contains at least
  \inlineocl{invalid} and \inlineocl{null} (the type \inlineocl{Void}
  contains only these instances). The logic is consequently
  four-valued, and there is a \inlineocl{null}-element in the type
  \inlineocl{Set(A)}.
\item It is a strongly typed language in the Hindley-Milner tradition.
  We assume that a pre-process eliminates all implicit conversions due
  to subtyping by introducing explicit casts (\eg,
  \inlineocl{oclAsType()}). The details of such a pre-processing are
  described in~\cite{brucker:interactive:2007}.  Casts are semantic
  functions, typically injections, that may convert data between the
  different Featherweight OCL types.
\item All objects are represented in an object universe in the \holocl
  tradition~\cite{brucker.ea:extensible:2008-b}. The universe
  construction also gives semantics to type casts, dynamic type
  tests, as well as functions such as \inlineocl{oclAllInstances()},
  or \inlineocl{oclIsNew()}.
\item Featherweight OCL types may be arbitrarily nested:
  \inlineocl|Set{Set{1,2}} = Set{Set{2,1}}| is legal and true.
\item For demonstration purposes, the set type in Featherweight OCL
  may be infinite, allowing infinite quantification and a constant
  that contains the set of all Integers.  Arithmetic laws like
  commutativity may therefore be expressed in OCL itself.  The
  iterator is only defined on finite sets.
\item It supports equational reasoning and congruence reasoning, but
  this requires a differentiation of the different equalities like
  strict equality, strong equality, meta-equality (HOL). Strict
  equality and strong equality require a subcalculus, ``cp'' (a
  detailed discussion of the different equalities as well as the
  subcalculus ``cp''---for three-valued OCL 2.0---is given
  in~\cite{brucker.ea:semantics:2009}), which is nasty but can be
  hidden from the user inside tools.
\end{enumerate}

\section{The Theory Organization}
The semantic theory is organized in a quite conventional manner in
three layers. The first layer, called the \emph{denotational
  semantics} comprises a set of definitions of the operators of the
language.  Presented as \emph{definitional axioms} inside
Isabelle/HOL, this part assures the logically consistency of the
overall construction. The second layer, called \emph{logical layer},
is derived from the former and centered around the notion of validity
of an OCL formula $P$ for a state-transition from pre-state $\sigma$
to post-state $\sigma'$, validity statements were written $(\sigma,
\sigma') \models P$.  The third layer, called \emph{algebraic layer},
also derived from the former layers, tries to establish a number of
algebraic laws of the form $P = P'$; such laws are amenable to
equational reasoning and also help for automated reasoning and
code-generation.

For space reasons, we will restrict ourselves in this paper to a few
operators and make a traversal through all three layers in order to
give a high-level description of our formalization.  Especially, the
details of the semantic construction for sets and the handling of
objects and object universes were excluded from a presentation here.

\subsection{Denotational Semantics}
 OCL is composed of
 \begin{enumerate}
 \item operators on built-in data structures such as Boolean, Integer or Set(A),
 \item operators of the user-defined data-model such as accessors,
   type-casts and tests, and
 \item user-defined, side-effect-free methods.
 \end{enumerate}
 Conceptually, an OCL expression in general and Boolean expressions in
 particular (\ie, \emph{formulae}) that depends on the pair $(\sigma,
 \sigma')$ of pre-and post-state.  The precise form of states is
 irrelevant for this paper (compare~\cite{brucker.ea:ocl-null:2009})
 and will be left abstract in this presentation. We construct in
 Isabelle a type-class $\TCnull$ that contains two distinguishable
 elements $\HolBot$ and $\HolNull$. Any type of the form
 $\up{(\up{\alpha})}$ is an instance of this type-class with $\HolBot
 \equiv \isasymbottom$ and $\HolNull \equiv \lift{\isasymbottom}$.
Now, any OCL type can be represented by an HOL type of the form:
\begin{equation*}
  \V{}{\alpha} \defeq \state{} \times \state{} \to \alpha \ofType \TCnull \mi{.}
\end{equation*}
On this basis, we define $\V{}{\up{(\up{\HolBoolean })}}$ as the HOL
type for the OCL type $\mocl{Boolean}$ by and define:
\begin{gather*}
\begin{alignedat}{3}
I\semantics{\mocl{invalid}\ofType V(\alpha)} \tau &\equiv \HolBot &
I\semantics{\mocl{null}\ofType V(\alpha)}  \tau    &\equiv \HolNull\\
I\semantics{\mocl{true}\ofType\mocl{Boolean}} \tau &=\lfloor\lfloor
\HolTrue\rfloor\rfloor &
I\semantics{\mocl{false}} \tau &= \lfloor\lfloor\HolFalse\rfloor\rfloor\\
\end{alignedat}\\
I\semantics{X\mocl{.oclIsUndefined()}} \tau =
    (\HolIf I\semantics{X}\tau \in \{\HolBot, \HolNull\} \HolThen I\semantics{\mocl{true}}\tau \HolElse I\semantics{\mocl{false}}\tau)\\
 I\semantics{X\mocl{.oclIsInvalid()}} \tau =
    (\HolIf I\semantics{X}\tau = \HolBot \HolThen I\semantics{\mocl{true}}\tau \HolElse I\semantics{\mocl{false}}\tau)
\end{gather*}
where $I\semantics{E}$ is the semantic interpretation function
commonly used in mathematical textbooks and $\tau$ stands for pairs of
pre- and post state $(\sigma, \sigma')$. Due to the used style of
semantic representation (a shallow embedding) $I$ is in fact
superfluous and defined semantically as the identity; in Isabelle
theories, it is usually left out in definitions to pave the way for
Isabelle to checks that the underlying equations are axiomatic
definitions and therefore logically safe. For reasons of conciseness,
we will write $\delta~X$ for $\mocl{not}\;X\mocl{.oclIsUndefined()}$
and $\upsilon~X$ for $\mocl{not}\;X\mocl{.oclIsInvalid()}$ throughout
this paper.


On this basis, one can define the core logical operators $\mocl{not}$
and $\mocl{and}$ as follows:
\begin{gather*}
  \begin{array}{ll}
    I\semantics{\mocl{not}\; X}  \tau
    &=  (\HolCase I\semantics{X} \tau  \HolOf\\
    &\quad\begin{array}{ll}
                     ~ \bottom                    &\Rightarrow  \bottom \\
                     | \lfloor  \bottom  \rfloor  &\Rightarrow  \lfloor  \bottom  \rfloor  \\
                     | \lfloor \lfloor  x \rfloor \rfloor  &\Rightarrow  \lfloor \lfloor  \lnot  x \rfloor \rfloor )\\
                   \end{array}\\
   I\semantics{X\;\mocl{and}\; Y}  \tau
    &=  (\HolCase I\semantics{X} \tau  \HolOf\\
    &\quad\begin{array}{ll}
      ~ \bottom                    &\Rightarrow
      (\HolCase I\semantics{Y} \tau  \HolOf\\
      &\quad\begin{array}{ll}
                     ~ \bottom                    &\Rightarrow  \bottom \\
                     | \lfloor  \bottom  \rfloor  &\Rightarrow  \bottom  \\
                     | \lfloor \lfloor  \HolTrue \rfloor \rfloor
                     &\Rightarrow  \bottom\\
                     | \lfloor \lfloor  \HolFalse \rfloor \rfloor
                     &\Rightarrow  \lfloor \lfloor  \HolFalse \rfloor \rfloor )\\
                   \end{array}
      \\
                     | \lfloor  \bottom  \rfloor  &\Rightarrow
      (\HolCase I\semantics{Y} \tau  \HolOf\\
      &\quad\begin{array}{ll}
                     ~ \bottom                    &\Rightarrow
                     \bottom \\
                     | \lfloor  \bottom  \rfloor  &\Rightarrow  \lfloor
                     \bottom \rfloor \\
                     | \lfloor \lfloor  \HolTrue \rfloor \rfloor
                     &\Rightarrow  \lfloor \bottom\rfloor\\
                     | \lfloor \lfloor  \HolFalse \rfloor \rfloor
                     &\Rightarrow  \lfloor \lfloor  \HolFalse \rfloor \rfloor )\\
                   \end{array}
      \\
                     | \lfloor \lfloor  \HolTrue \rfloor \rfloor  &\Rightarrow
      (\HolCase I\semantics{Y} \tau  \HolOf\\
      &\quad\begin{array}{ll}
                     ~ \bottom                    &\Rightarrow
                     \bottom \\
                     | \lfloor  \bottom  \rfloor  &\Rightarrow  \lfloor
                     \bottom \rfloor \\
                     | \lfloor \lfloor y \rfloor \rfloor
                     &\Rightarrow  \lfloor \lfloor  y \rfloor \rfloor )\\
                   \end{array}
      \\
                     | \lfloor \lfloor  \HolFalse \rfloor \rfloor
                     &\Rightarrow   \lfloor \lfloor  \HolFalse \rfloor
                     \rfloor )\\
                   \end{array}\\
  \end{array}
\end{gather*}
These non-strict operations were used to define the other logical connectives in the
usual classical way: $X\; \mocl{or}\; Y \equiv (\mocl{not}\; X)\;
\mocl{and}\; (\mocl{not}\; Y)$ or
$X\;\mocl{implies}\;Y \equiv (\mocl{not}\; X)\;\mocl{or}\; Y$.

The default semantics for an OCL library operator is strict
semantics; this means that the result of an operation $f$ is
\inlineisar+invalid+ if one of its arguments is \inlineisar+invalid+.
For a semantics comprising \inlineisar+null+, we suggest to stay
conform to the standard and define the addition for integers as
follows:
 \begin{gather*}
   \begin{array}{rl}
   I\semantics{x \mocl{+} y}\tau  = &\HolIf I\semantics{\delta ~ x}\tau =\lfloor \lfloor \HolTrue\rfloor \rfloor  \land   I\semantics{\delta  ~ y}\tau =\lfloor \lfloor \HolTrue\rfloor \rfloor \\
                &\HolThen \lfloor \lfloor \lceil \lceil I\semantics{x}\tau \rceil \rceil  + \lceil \lceil I\semantics{y}\tau \rceil \rceil \rfloor \rfloor\\
                &\HolElse \bottom
   \end{array}
 \end{gather*}
 where the operator ``\mocl{+}'' on the left-hand
 side of the equation denotes the OCL addition of type
 $[\V{}{\up{(\up{\HolInteger})}}, \V{}{\up{(\up{\HolInteger})}}] \Rightarrow
 \V{}{\up{(\up{\HolInteger})}}$ while the ``$+$'' on the right-hand side of
 the equation of type $[\HolInteger,\HolInteger]\Rightarrow
 \HolInteger$ denotes the integer-addition from the HOL library.

\subsection{Logical Layer}
The topmost goal of the logic for OCL is to define the \emph{validity statement}:
\begin{equation*}
   (\sigma, \sigma') \isasymMathOclValid P\mi{,}
\end{equation*}
where $\sigma$ is the pre-state and $\sigma'$ the post-state of the
underlying system and $P$ is a formula.
Informally, a formula $P$ is valid if and only if its evaluation in
$(\sigma, \sigma')$ (\ie, $\tau$ for short) yields true. Formally this means:
\begin{equation*}
\tau \isasymMathOclValid P \equiv \bigl(I\semantics{P} \tau =  \lfloor \lfloor \HolTrue  \rfloor \rfloor \bigr)\mi{.}
\end{equation*}
On this basis, classical, two-valued inference rules can be established for
reasoning over the logical connective, the different notions of equality,
definedness and validity. Generally speaking, rules over logical validity can
relate bits and pieces in various OCL terms and allow---via strong
logical equality discussed below---the replacement
of semantically equivalent sub-expressions. The core inference rules are:
\begin{gather*}
  \begin{array}{lccr}
  \tau \models \mocl{true}&
  \lnot(\tau \models \mocl{false})&
  \lnot(\tau \models \mocl{invalid})&
  \lnot(\tau \models \mocl{null})
\end{array}\\
  \tau \models \mocl{not}\; P \Longrightarrow \tau \lnot \models P\\
\begin{array}{lcr}
  \tau \models P \;\mocl{and}\; Q \Longrightarrow \tau \models P&\quad&
  \tau \models P \;\mocl{and}\; Q \Longrightarrow \tau \models Q
  \end{array}\\
  \tau \models P \Longrightarrow
     (\mocl{if}\; P \;\mocl{then}\; B_1 \;\mocl{else}\; B_2 \;\mocl{endif})\tau = B_1 \tau\\
  \tau \models \mocl{not}\; P \Longrightarrow
       (\mocl{if}\; P \;\mocl{then}\; B_1 \;\mocl{else}\; B_2 \;\mocl{endif})\tau = B_2 \tau\\
       \begin{array}[lcr]{lcr}
  \tau \models P \Longrightarrow \tau \models \delta P &\qquad&
  \tau \models (\delta X) \Longrightarrow \tau \models \upsilon X
       \end{array}
\end{gather*}
By the latter two properties it can be inferred that any valid
property $P$ (so for example: a valid invariant) is actually defined,
which allows to infer for terms composed by strict operations that
their arguments and finally the variables occurring in it are valid or
defined.

We propose to distinguish the \emph{strong logical equality} (written
$\_ \isasymMathOclStrongEq \_$), which follows the general principle
that ``equals can be replaced by equals,'' from the \emph{strict
  referential equality} (written $\_ \isasymMathOclStrictEq \_$),
which is an object-oriented concept that attempts to approximate and
to implement the former.  Strict referential equality, which is the
default in the OCL language and is written simply \inlineocl|_ = _|
in the standard, is an overloaded concept and has to be defined for
each OCL type individually; for objects resulting from class
definitions, it is implemented by simply comparing the references to
the objects. In contrast, strong logical equality is a polymorphic
concept which is defined once and for all by:
\begin{gather*}
  I\semantics{X \triangleq  Y}  \tau  \equiv  \lfloor \lfloor
  I\semantics{X}  \tau=
  I\semantics{Y} \tau  \rfloor \rfloor
\shortintertext{It enjoys nearly the laws of a congruence:}
\tau \models (x \triangleq x)\\
\tau \models (x \triangleq y) \Longrightarrow \tau \models (y \triangleq x)\\
\tau \models (x \triangleq y) \Longrightarrow \tau \models (y \triangleq z) \Longrightarrow \tau \models (x \triangleq z)\\
\HolOclCp P \Longrightarrow \tau \models (x \triangleq y) \Longrightarrow \tau \models (P\ap x) \Longrightarrow \tau \models (P\ap y)
\end{gather*}
where the predicate $\HolOclCp$ stands for \emph{context-passing}, a
property that is characterized by $P(X)$ equals $\lambda \tau\spot
P(\lambda \_\spot X \tau) \tau$. It means that the state tuple $\tau =
(\sigma, \sigma')$ is passed unchanged from surrounding expressions to
sub-expressions. it is true for all pure OCL expressions (but not
arbitrary mixtures of OCL and HOL) in Featherweight OCL.  The
necessary side-calculus for establishing $\HolOclCp$ can be fully
automated.

The logical layer of the Featherweight OCL rules gives also a means
to convert an OCL formula living in its for-valued world into a
representation that is classically two-valued and can be processed by
standard SMT solvers such as CVC3~\cite{barrett.ea:cvc3:2007} or
Z3~\cite{moura.ea:z3:2008}. $\delta$-closure rules for all logical
connectives have the following format, \eg:
\begin{gather*}
\tau \models \delta x \Longrightarrow (\tau \models \ap\mocl{not}\ap x) = (\lnot (\tau \models x))\\
\tau \models \delta x \Longrightarrow \tau \models \delta y \Longrightarrow (\tau \models x \ap\mocl{and}\ap y) = ( \tau \models x \land \tau \models y)\\
\begin{multlined}
\tau \models \delta x \Longrightarrow  \tau \models \delta y \\
\Longrightarrow (\tau \models (x \ap\mocl{implies}\ap y)) = ( (\tau \models x) \longrightarrow (\tau \models y))
\end{multlined}
\end{gather*}
Together with the general case-distinction
\begin{gather*}
    \tau \models \delta x \lor \tau \models x \triangleq \mocl{invalid} \lor \tau \models x \triangleq \mocl{null} \mi{,}
\end{gather*}
which is possible for any OCL type, a case distinction on the variables in a formula can be performed; due
to strictness rules, formulae
containing somewhere a variable $x$ that is known to be
$\mocl{invalid}$ or $\mocl{null}$ reduce usually quickly to
contradictions.  For example, we can infer from an invariant $\tau
\models x \isasymMathOclStrictEq y \mocl{-} 3$ that we have actually
$\tau \models x \isasymMathOclStrictEq y \mocl{-} 3 \land \tau \models
\delta x \land \tau \models \delta y$.  We call the latter formula the
$\delta$-closure of the former.  Now, we can convert a formula like
$\tau \models x \mocl{>} 0 ~\text{or} 3 \mocl{*} y \mocl{>} x\mocl{*}x$ into
the equivalent formula $\tau \models x > 0 \lor \tau \models 3
\mocl{*} y \mocl{>} x\mocl{*}x$ and thus internalize the OCL-logic
into a classical (and more tool-conform) logic. This works---for the
price of a potential, but due to the usually ``rich'' $\delta$-closures
of invariants rare---exponential blow-up of the formula for all OCL
formulas.

\subsection{Algebraic Layer}
Based on the logical layer, we build a system with simpler rules which
are amenable to automated reasoning. We restrict ourselves to pure
equations on OCL expressions, where the used equality is the
meta-(HOL-)equality.

Our denotational definitions on \inlineocl+not+ and \inlineocl+and+
can be re-formulated in the following ground equations:
\begin{gather*}
  \begin{aligned}
  \upsilon\; \mocl{invalid} &= \mocl{false}&
  \upsilon\; \mocl{null} &= \mocl{true}\\
  \upsilon\; \mocl{true} &= \mocl{true}&
  \upsilon\; \mocl{false} &= \mocl{true}\\
\end{aligned}\\[.04\baselineskip]
\begin{aligned}
  %
  \delta\; \mocl{invalid} &= \mocl{false}&
  \delta\; \mocl{null} &= \mocl{false}\\
  \delta\; \mocl{true} &= \mocl{true}&
  \delta\; \mocl{false} &= \mocl{true}\\
\end{aligned}\\[.04\baselineskip]
\begin{aligned}
  %
  \mocl{not}\; \mocl{invalid} &= \mocl{invalid}&
  \mocl{not}\; \mocl{null} &= \mocl{null}\\
  \mocl{not}\; \mocl{true} &= \mocl{false}&
  \mocl{not}\; \mocl{false} &= \mocl{true}\\
\end{aligned}\\[.04\baselineskip]
\begin{aligned}
  %
  (\mocl{null} \;\mocl{and}\; \mocl{true}) &= \mocl{null}&
  (\mocl{null} \;\mocl{and}\; \mocl{false}) &= \mocl{false}\\
  (\mocl{null} \;\mocl{and}\; \mocl{null}) &= \mocl{null}&
  (\mocl{null} \;\mocl{and}\; \mocl{invalid}) &= \mocl{invalid}\\
\end{aligned}\\[.04\baselineskip]
\begin{aligned}
  %
  (\mocl{false} \;\mocl{and}\; \mocl{true}) &= \mocl{false}&
  (\mocl{false} \;\mocl{and}\; \mocl{false}) &= \mocl{false}\\
  (\mocl{false} \;\mocl{and}\; \mocl{null}) &= \mocl{false}&
  (\mocl{false} \;\mocl{and}\; \mocl{invalid}) &= \mocl{false}\\
\end{aligned}\\[.04\baselineskip]
\begin{aligned}
  %
  (\mocl{true} \;\mocl{and}\; \mocl{true}) &= \mocl{true}&
  (\mocl{true} \;\mocl{and}\; \mocl{false}) &= \mocl{false}\\
  (\mocl{true} \;\mocl{and}\; \mocl{null}) &= \mocl{null}&
  (\mocl{true} \;\mocl{and}\; \mocl{invalid}) &= \mocl{invalid}
\end{aligned}\\[.04\baselineskip]
\begin{aligned}
  (\mocl{invalid} \;\mocl{and}\; \mocl{true}) &= \mocl{invalid} \\
  (\mocl{invalid} \;\mocl{and}\; \mocl{false}) &= \mocl{false}\\
  (\mocl{invalid} \;\mocl{and}\; \mocl{null}) &= \mocl{invalid} \\
  (\mocl{invalid} \;\mocl{and}\; \mocl{invalid}) &= \mocl{invalid}\\
\end{aligned}
\shortintertext{On this core, the structure of a conventional lattice arises:}
  \begin{aligned}
    X \;\mocl{and}\; X &= X        &     X \;\mocl{and}\; Y &= Y \;\mocl{and}\; X
  \end{aligned}\\
  \begin{aligned}
    \mocl{false} \;\mocl{and}\; X &= \mocl{false} &
    X \;\mocl{and}\; \mocl{false} &= \mocl{false}  \\
    \mocl{true} \;\mocl{and}\; X  &= X &
    X \;\mocl{and}\; \mocl{true} &= X
  \end{aligned}\\
  \begin{aligned}
             X \;\mocl{and}\; (Y \;\mocl{and}\; Z) &= X \;\mocl{and}\; Y \;\mocl{and}\; Z
  \end{aligned}
\end{gather*}
as well as the dual equalities for \inlineocl{or} and the De Morgan
rules.  This wealth of algebraic properties makes the understanding of
the logic easier as well as automated analysis possible: it allows
for, for example, computing a DNF of invariant systems (by
clever term-rewriting techniques) which are a prerequisite for
$\delta$-closures.

The above equations explain the behavior for the most-important
non-strict operations. The clarification of the exceptional behaviors
is of key-importance for a semantic definition the standard and the
major deviation point from
\holocl~\cite{brucker.ea:hol-ocl:2008,brucker.ea:hol-ocl-book:2006},
to Featherweight OCL as presented here.  The
standard expresses at many places that most operations are strict,
\ie, enjoy the properties (exemplary for \inlineocl{_ + _}):
\begin{gather*}
  \begin{aligned}
  \mocl{invalid}\;\mocl{+}\; x &= \mocl{invalid}&
  \mocl{x} \;\mocl{+}\; \mocl{invalid} &= \mocl{invalid}\\
  x \;\mocl{+}\; \mocl{null} &= \mocl{invalid}&
  \mocl{null} \;\mocl{+}\; x &= \mocl{invalid}
  \end{aligned}\\
  \mocl{null.asType(}X\mocl{)} = \mocl{invalid}
\shortintertext{besides ``classical'' exceptional behavior:}
  \begin{aligned}
    \mocl{1 / 0} &= \mocl{invalid} \quad &\quad  \mocl{1 / null} &= \mocl{invalid}
  \end{aligned}\\
    \mocl{null->isEmpty()}=\mocl{true}
\end{gather*}

Moreover, there is also the proposal to use \inlineocl+null+ as a kind
of ``don't know'' value for all strict operations, not only in the
semantics of the logical connectives.  Expressed in algebraic
equations, this semantic alternative (this is \emph{not}
Featherweight OCL at present) would boil down to:
\begin{gather*}
  \begin{aligned}
    \mocl{invalid} \;\mocl{+}\; x &= \mocl{invalid} &
    x \;\mocl{+}\; \mocl{invalid} &= \mocl{invalid}\\
    x \;\mocl{+}\; \mocl{null} &= \mocl{null}&
    \mocl{null} \;\mocl{+}\; x &= \mocl{null}\\
    1 / 0 &= \mocl{invalid}&
    1 / \mocl{null} &= \mocl{null}
  \end{aligned}\\
  \begin{aligned}
    \mocl{null->isEmpty()}&=\mocl{null}&
    \mocl{null.asType(}X\mocl{)} &= \mocl{null}
\end{aligned}
\end{gather*}

While this is logically perfectly possible, while it can be argued
that this semantics is ``intuitive,'' and although we do not expect a
too heavy cost in deduction when computing $\delta$-closures, we
object that there are other, also ``intuitive'' interpretations that
are even more wide-spread: In classical spreadsheet programs, for
example, the semantics tends to interpret \inlineocl+null+
(representing empty cells in a sheet) as the neutral element of the
type, so \inlineocl{0} or the empty string, for example.\footnote{In
  spreadsheet programs the interpretation of \inlineisar+null+ varies
  from operation to operation; \eg, the \inlineocl+average+
  function treats \inlineocl+null+ as non-existing value and not as
  \inlineocl{0}.}  This semantic alternative (this is
\emph{not} Featherweight OCL at present) would yield:
\begin{gather*}
  \begin{aligned}
    \mocl{invalid} \;\mocl{+}\; x &= \mocl{invalid} &
    x \;\mocl{+}\; \mocl{invalid} &= \mocl{invalid}\\
    x \;\mocl{+}\; \mocl{null} &= x&
    \mocl{null} \;\mocl{+}\; x &= x\\
    1 / 0 &= \mocl{invalid}&
    1 / \mocl{null} &= \mocl{invalid}
  \end{aligned}\\
  \begin{aligned}
    \mocl{null->isEmpty()}&=\mocl{true}&
    \mocl{null.asType(}X\mocl{)} &= \mocl{invalid}
\end{aligned}
\end{gather*}

Algebraic rules are also the key for execution and compilation
of Featherweight OCL expressions. We derived, \eg:
\begin{gather*}
\delta\; \mocl{Set\{\}} = \mocl{true}\\
\delta\; (X\mocl{->including(}x\mocl{)}) = \delta X \;\mocl{and}\;
\delta x\\
\begin{aligned}
\mocl{Set\{\}->includes(}x\mocl{)} = (\mocl{if}\; \upsilon\; x\; &\mocl{then false}\\
&\mocl{else invalid endif})
\end{aligned}\\
\begin{multlined}
  {(X\mocl{->including(}x\mocl{)->includes(}y\mocl{)})=}\\
  \mbox{\hspace{3.2cm}}\qquad{\begin{aligned}
   (&\mocl{if}\; \delta\; X\\
  &\mocl{then}\;
\begin{array}[t]{l}
\mocl{if}\; x \doteq y\\
\mocl{then true} \\
\mocl{else} X\mocl{->includes(}y\mocl{)}\\
\mocl{endif}
  \end{array}\\
&\mocl{else invalid} \\
         &\mocl{endif})
  \end{aligned}}
\end{multlined}
\end{gather*}
As \inlineocl+Set{1,2}+ is only syntactic sugar for
\begin{ocl}
  Set{}->including(1)->including(2)
\end{ocl}
an expression like \inlineocl+Set{1,2}->includes(null)+ becomes
automatically decidable in Featherweight OCL by a combination of
rewriting and code-generation and execution. The generated
documentation from the theory files can thus be enriched by numerous
``test-statements'' like:
\begin{isar}[mathescape]
value  "\<tau> \<Turnstile> ($\mathtt{Set\{Set\{2,null\}\}}$ \<doteq> $\;\mathtt{Set\{Set\{null,2\}\}}$)"
\end{isar}
which have been machine-checked and which present a high-level and in
our opinion fairly readable information for OCL tool manufactures and
users.





\section{A Machine-checked Annex A}
\begin{figure*}[tb]
  \mbox{}\hfill
  \subfloat%
  [The Isabelle jEdit environment. ]%
  {\label{fig:jedit} \includegraphics[height=6.2cm]{jedit}}%
  \hfill%
  \hfill%
    \subfloat[The generated formal document.]%
    {\label{fig:pdf} \includegraphics[height=6.2cm]{pdf}}
    \hfill\mbox{}
  \caption{Generating documents with guaranteed  syntactical and
    semantical consistency.}
  \label{fig:gener-docum-where}
\end{figure*}
Isabelle, as a framework for building formal
tools~\cite{wenzel.ea:building:2007}, provides the means for
generating \emph{formal documents}.  With formal documents (such as
the one you are currently reading) we refer to documents that are
machine-generated and ensure certain formal guarantees. In particular,
all formal content (\eg, definitions, formulae, types) are checked for
consistency during the document generation.

For writing documents, Isabelle supports the embedding of informal
texts using a \LaTeX-based markup language within the theory files. To
ensure the consistency, Isabelle supports to use, within these
informal texts, \emph{antiquotations} that refer to the formal parts and
that are checked while generating the actual document as
PDF. For example, in an informal text, the antiquotation
\inlineisar|@{$\text{thm}$ "not_not"}| will instruct Isabelle to lock-up the
(formally proven) theorem of name \inlineisar"ocl_not_not" and to replace
the antiquotation with the actual theorem, \ie,
\inlineocl{not (not x) $=$ x}.

\autoref{fig:gener-docum-where}
illustrates this approach: \autoref{fig:jedit} shows the jEdit-based
development environment of Isabelle with an excerpt of one of the core
theories of Featherweight OCL. \autoref{fig:pdf} shows the generated
PDF document where all antiquotations are replaced. Moreover,
the document generation tools allows for defining syntactic sugar as
well as skipping technical details of the formalization.


Thus, applying the Featherweight OCL approach to writing an updated
Annex A that provides a formal semantics of the most fundamental
concepts of OCL would ensure
\begin{enumerate}
\item that all formal context is syntactically correct and well-typed,
  and
\item all formal definitions and the derived logical rules are
  semantically consistent.
\end{enumerate}
Overall, this would contribute to one of the main goals of the OCL 2.5
RFP, as discussed at the OCL meeting in
Aachen~\cite{brucker.ea:summary-aachen:2013}.
 
%%% Local Variables: 
%%% mode: latex
%%% TeX-master: "root"
%%% End:

\part{A Proposal for Formal Semantics of OCL 2.5}

\input{session}

%%% Local Variables: 
%%% mode: latex
%%% TeX-master: "root"
%%% End:

\part{Conclusion}

\chapter{Conclusion}

\section{Lessons Learned}
While our paper and pencil arguments, given
in~\cite{brucker.ea:ocl-null:2009}, turned out to be essentially
correct, there had also been a lesson to be learned: If the logic is
not defined as a Kleene-Logic, having a structure similar to a
\acf{cpo}, reasoning becomes complicated: several important algebraic laws break down
which makes reasoning in \OCL inherent messy and a semantically clean
compilation of \OCL formulae to a two-valued presentation, that is
amenable to animators like KodKod~\cite{torlak.ea:kodkod:2007} or
\acs{smt}-solvers like Z3~\cite{moura.ea:z3:2008} completely
impractical. Concretely, if the expression \inlineocl{not(null)}
is defined \inlineocl{invalid} (as is the case in the
present standard~\cite{omg:ocl:2012}), than standard involution does
not hold, \ie, \inlineocl{not(not(A))} = \inlineocl{A} does not hold
universally. Similarly, if
% \begin{ocl}
%      null and null  
% \end{ocl}
\inlineocl{null and null} is \inlineocl{invalid}, then not even
idempotence \inlineocl{X and X} = \inlineocl{X} holds. We strongly argue in favor of a lattice-like
organization, where \inlineocl{null} represents ``more information''
than \inlineocl{invalid} and the logical operators are monotone with
respect to this semantical ``information ordering.''
%(as in classical Scott \acs{cpo}'s).

Featherweight \OCL makes these two deviations from the standard,
builds all logical operators on Kleene-\inlineocl{not} and
Kleene-\inlineocl{and}, and shows that the entire construction of our
paper ``Extending \OCL with
Null-References''~\cite{brucker.ea:ocl-null:2009} is then correct, and
the \acs{dnf}-normaliation as well as $\delta$-closure laws (necessary
for a transition into a two-valued presentation of \OCL specifications
ready for interpretation in \acs{smt} solvers
(see~\cite{brucker.ea:ocl-testing:2010} for details) are valid in
Featherweight \OCL.

\section{Conclusion and Future Work}
Featherweight \OCL concentrates on formalizing the semantics of a core
subset of \OCL in general and in particular on formalizing the
consequences of a four-valued logic (\ie, \OCL versions that support,
besides the truth values \inlineocl{true} and \inlineocl{false} also
the two exception values \inlineocl{invalid} and \inlineocl{null}).

In the following, we outline the necessary steps for turning
Featherweight \OCL into a fully fledged tool for \OCL, \eg, similar to
\holocl as well as for supporting test case generation similar to
\acs{hol}-TestGen~\cite{brucker.ea:hol-testgen:2009}.  There are
essentially five extensions necessary:
\begin{itemize}
\item extension of  the library to support all \OCL data types, \eg,
  \inlineocl{Sequence(T)}, \inlineocl{OrderedSet(T)}. % This is
  % essentially a student-project given the sample proofs already there,
  % that have to be adapted to similar types.
  This formalization of the \OCL standard library can be used for
  checking the consistency of the formal semantics (known as ``Annex
  A'') with the informal and semi-formal requirements in the normative
  part of the \OCL standard.
\item development of a compiler that compiles a textual or \acs{case}
  tool representation (\eg, using \acs{xmi} or the textual syntax of
  the \acs{use} tool~\cite{richters:precise:2002}) of class
  models. Such compiler could also generate the necessary casts when
  converting standard \OCL to Featherweight \OCL as well as providing
  ``normalizations'' such as converting multiplicities of class
  attributes to into \OCL class invariants.
\item a setup for translating Featherweight \OCL into a two-valued
  representation as described
  in~\cite{brucker.ea:ocl-testing:2010}. As, in real-world scenarios,
  large parts of \acs{uml}/\acs{ocl} specifications are defined (\eg,
  from the default multiplicity \inlineocl{1} of an attributes
  \inlineocl{x}, we can directly infer that for all valid states
  \inlineocl{x} is neither \inlineocl{invalid} nor \inlineocl{null}),
  such a translation enables an efficient test case generation
  approach.
\item a setup in Featherweight \OCL of the Nitpick
  animator~\cite{blanchette.ea:nitpick:2010}. It remains to be shown
  that the standard, Kodkod~\cite{torlak.ea:kodkod:2007} based
  animator in Isabelle can give a similar quality of animation as the
  OCLexec Tool~\cite{krieger.ea:generative:2010}
\item a code-generator setup for Featherweight \OCL for Isabelle's
  code generator. For example, the Isabelle code generator supports
  the generation of F\#, which would allow to use \acs{ocl}
  specifications for testing arbitrary .net-based applications.
\end{itemize}
The first two extensions are sufficient to provide a formal proof
environment for \OCL 2.3 similar to \holocl while the remaining
extensions are geared towards increasing the degree of proof
automation and usability as well as providing a tool-supported test
methodology for \acs{uml}/\acs{ocl}.


Our work shows that developing a machine-checked formal semantics of
recent \acs{ocl} standards still reveals significant
inconsistencies---even though this type of research is not new. In
fact, we started our work already with the 1.x series of
\acs{ocl}. The reasons for this ongoing consistency problems of
\acs{ocl} standard are manifold. For example, the consequences of
adding an additional exception value to \OCL 2.2 are widespread across
the whole language and many of them are also quite subtle. Here, a
machine-checked formal semantics is of great value, as one is forced
to formalize all details and subtleties.  Moreover, the
standardization process of the \acs{omg}, in which standards (\eg, the
\acs{uml} infrastructure and the \acs{ocl} standard) that need to be
aligned closely are developed quite independently, are prone to ad-hoc
changes that attempt to align these standards. And, even worse,
updating a standard document by voting on the acceptance (or
rejection) of isolated text changes does not help either. Here, a tool
for the editor of the standard that helps to check the consistency of
the whole standard after each and every modifications can be of great
value as well.



% optional bibliography
\bibliographystyle{abbrv}
\bibliography{adb-long,root}
\listofacronyms{}

\end{document}

%%% Local Variables:
%%% mode: latex
%%% TeX-master: t
%%% End:

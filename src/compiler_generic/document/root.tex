\documentclass[fontsize=11pt,paper=a4,open=right,twoside,abstract=true]{scrreprt}
\usepackage[T1]{fontenc}
\usepackage[utf8]{inputenc}
\usepackage{lmodern}
\usepackage{textcomp}
\usepackage[english]{babel}
%\usepackage[draft]{fixme}
\usepackage{graphicx}
\usepackage[numbers, sort&compress, sectionbib]{natbib}
\usepackage{amssymb}
\usepackage{versions}
\usepackage{isabelle,isabellesym}
\usepackage{units}
%\usepackage{eurosym}
\IfFileExists{railsetup.sty}{\usepackage{railsetup}}{}
\usepackage{titletoc}

%%%%%%%%%%%%%%%%%%%%%%%%%%%%%%%%%%%%%%%%%%%%%%%%%%%%%%%%%%%%%%%%%%%%%%%%%%%%%%%
% short vs. long version

%%%% Short Version:
\includeversion{short}
\excludeversion{extended}

%%%% Extended Version:
%\excludeversion{short}
%\includeversion{extended}

%%%% Misc.:
\newenvironment{shortspace}[1]{}{} %\processifversion{short}{\vspace{#1}}

%%%%%%%%%%%%%%%%%%%%%%%%%%%%%%%%%%%%%%%%%%%%%%%%%%%%%%%%%%%%%%%%%%%%%%%%%%%%%%%
% command

\graphicspath{{data/},{figures/}}

%%

\newenvironment{matharray}[1]{\[\begin{array}{#1}}{\end{array}\]} % from 'iman.sty'
\newcommand{\indexdef}[3]%
{\ifthenelse{\equal{}{#1}}{\index{#3 (#2)|bold}}{\index{#3 (#1\ #2)|bold}}} % from 'isar.sty'

%%

\newcommand\inputif[1]{\IfFileExists{./#1}{\input{#1}}{}}

%%%%%%%%%%%%%%%%%%%%%%%%%%%%%%%%%%%%%%%%%%%%%%%%%%%%%%%%%%%%%%%%%%%%%%%%%%%%%%%
% fix for package declaration to be at the end
\usepackage[pdfpagelabels, pageanchor=false, plainpages=false]{hyperref}

%%%%%%%%%%%%%%%%%%%%%%%%%%%%%%%%%%%%%%%%%%%%%%%%%%%%%%%%%%%%%%%%%%%%%%%%%%%%%%%
% document

\urlstyle{rm}
\isabellestyle{it}

\begin{document}

\title{A Meta-Model for the Isabelle API}
\author{%
  \href{https://www.lri.fr/~tuong/}{Fr\'ed\'eric Tuong}
  \and
  \href{https://www.lri.fr/~wolff/}{Burkhart Wolff}}
\publishers{%
  \mbox{LRI, Univ Paris Sud, CNRS, CentraleSup\'elec, Universit\'e Paris-Saclay} \\
  b\^at. 650 Ada Lovelace, 91405 Orsay, France \texorpdfstring{\\}{}
    \href{mailto:"Frederic Tuong"
    <frederic.tuong@lri.fr>}{frederic.tuong@lri.fr} \hspace{4.5em}
    \href{mailto:"Burkhart Wolff"
    <burkhart.wolff@lri.fr>}{burkhart.wolff@lri.fr} \\[2em]
  %
  IRT SystemX\\
  8 av.~de la Vauve, 91120 Palaiseau, France \texorpdfstring{\\}{}
    \href{mailto:"Frederic Tuong"
    <frederic.tuong@irt-systemx.fr>}{frederic.tuong@irt-systemx.fr} \quad
    \href{mailto:"Burkhart Wolff"
    <burkhart.wolff@irt-systemx.fr>}{burkhart.wolff@irt-systemx.fr}
}

\maketitle

\begin{abstract}
We represent a theory \emph{of} (a fragment of) Isabelle/HOL 
\emph{in} Isabelle/HOL. The purpose of this exercise is to write packages for
domain specific specifications constructs such as class models, B-machines, ...,
in HOL itself; the Isabelle code-generator can then be used to generate
tactic code.

Consequently, at present, the package is geared towards 
parsing, pretty-printing and code-generation to the Isabelle API's; 
it is at the moment not sufficiently rich for doing meta theory on 
Isabelle itself. Extensions in this direction are possible, though.

Moreover, the chosen fragment is fairly rudimentary. However, it should be 
easily adapted to one's needs if a package is written on top of it.
Nevertheless, the supported API contains types, terms, transformation of
global context like definitions and data-type declarations as well
as infrastructure for Isar-setups.

This theory is drawn from the Featherweight OCL\cite{brucker.ea:featherweight:2014} project where 
it is used to construct a package for object-oriented data-type theories
generated from UML class diagrams. The Featherweight OCL, for example, allows for 
both the direct execution of compiled tactic code by the Isabelle API
as well as the generation of \verb|.thy|-files for debugging purposes.

Gained experience from this project shows that the compiled code is sufficiently
efficient for practical purposes while being based on a formal \emph{model}
on which properties of the package can be proven such as termination of certain
transformations, correctness, etc.
\end{abstract}
\tableofcontents

\parindent 0pt\parskip 0.5ex

%%%%%%%%%%%%%%%%%%%%%%%%%%%%%%%%%%
%\input{session}

\part{A Meta-Model for the Isabelle API}
\chapter{Initialization}
\input{Init.tex}

\chapter{Defining Meta-Models}
\input{Meta_Pure.tex}
\input{Meta_SML.tex}
\input{Meta_Isabelle.tex}

\chapter{Parsing Meta-Models}
\input{Parser_init.tex}
\input{Parser_Pure.tex}

\chapter{Printing Meta-Models}
\input{Printer_init.tex}
\input{Printer_Pure.tex}
\input{Printer_SML.tex}
\input{Printer_Isabelle.tex}

\chapter{Main}
\inputif{Generator_dynamic.tex}

\part{A Toy Example}
\inputif{Toy_Library.tex}
\inputif{Design_deep.tex}
\inputif{Design_shallow.tex}

%%%%%%%%%%%%%%%%%%%%%%%%%%%%%%%%%%

\bibliographystyle{abbrvnat}
\bibliography{root}

%%%%%%%%%%%%%%%%%%%%%%%%%%%%%%%%%%

\appendix
\part{Appendix}
\chapter{Grammars of Commands}
\inputif{Rail.tex}

\end{document}

%%% Local Variables:
%%% mode: latex
%%% TeX-master: t
%%% End:

\documentclass[fontsize=11pt,paper=a4,open=right,twoside,abstract=true]{scrreprt}
\usepackage[T1]{fontenc}
\usepackage[utf8]{inputenc}
\usepackage{lmodern}
\usepackage{textcomp}
\usepackage[english]{babel}
%\usepackage[draft]{fixme}
%\usepackage{fmde-acronyms}
\usepackage{graphicx}
\usepackage[numbers, sort&compress, sectionbib]{natbib}
\usepackage[nocolortable, noaclist]{hol-ocl-isar}
\usepackage{amsthm}
\usepackage{mathtools}
\usepackage{units}
\usepackage{eurosym}
\usepackage{versions}
\usepackage{lstisar}
\IfFileExists{railsetup.sty}{\usepackage{railsetup}}{}
\usepackage{paralist}
%% \usepackage[
%% %final
%% submission
%% ]{anonymize} % [final]
\usepackage{tabu}
%\usepackage{prooftree}
\usepackage{tikz}
\usepackage{titletoc}

%%%%%%%%%%%%%%%%%%%%%%%%%%%%%%%%%%%%%%%%%%%%%%%%%%%%%%%%%%%%%%%%%%%%%%%%%%%%%%%
% short vs. long version

%%%% Short Version:
\includeversion{short}
\excludeversion{extended}

%%%% Extended Version:
%\excludeversion{short}
%\includeversion{extended}

%%%% Misc.:
\newenvironment{shortspace}[1]{}{} %\processifversion{short}{\vspace{#1}}

%%%%%%%%%%%%%%%%%%%%%%%%%%%%%%%%%%%%%%%%%%%%%%%%%%%%%%%%%%%%%%%%%%%%%%%%%%%%%%%
% command

\graphicspath{{data/},{figures/}}

%%

\newenvironment{matharray}[1]{\[\begin{array}{#1}}{\end{array}\]} % from 'iman.sty'
\newcommand{\indexdef}[3]%
{\ifthenelse{\equal{}{#1}}{\index{#3 (#2)|bold}}{\index{#3 (#1\ #2)|bold}}} % from 'isar.sty'

%%

\newcommand\inputif[1]{\IfFileExists{./#1}{\input{#1}}{}}

%%%%%%%%%%%%%%%%%%%%%%%%%%%%%%%%%%%%%%%%%%%%%%%%%%%%%%%%%%%%%%%%%%%%%%%%%%%%%%%
% fix for package declaration to be at the end
\usepackage[pdfpagelabels, pageanchor=false, plainpages=false]{hyperref}

%%%%%%%%%%%%%%%%%%%%%%%%%%%%%%%%%%%%%%%%%%%%%%%%%%%%%%%%%%%%%%%%%%%%%%%%%%%%%%%
% document

\begin{document}

\title{A Meta-Model for the Isabelle API}
\author{%
author
}

\maketitle

  \begin{abstract}
We represent a model in Isabelle/HOL allowing to write packages for
domain specific specifications constructs (such as class models, B-machines...)
in HOL itself. At present the package is geared towards parsing pretty-printing and
code-generation to the Isabelle API's; it is for the moment not sufficiently rich
for doing meta theory on Isabelle itself.
  \end{abstract}
\tableofcontents

\parindent 0pt\parskip 0.5ex

%%%%%%%%%%%%%%%%%%%%%%%%%%%%%%%%%%
\input{session}

\cite{nipkow.ea:isabelle:2002}
%%%%%%%%%%%%%%%%%%%%%%%%%%%%%%%%%%

\bibliographystyle{abbrv}
\bibliography{root}

\end{document}

%%% Local Variables:
%%% mode: latex
%%% TeX-master: t
%%% End:
